\documentclass[a4paper,12pt]{article}
\pagestyle{headings}

\begin{document}
\title{\textbf{Documentation of MIPS Single Cycle Processor}}
\author{\textsc{Kartik Patel}}
\date{}
\maketitle

\section{Introduction}
MIPS Single Cycle Processor is one of the most basic processor. It contains some basic hardwares like ALU, Adder, RegisterFile, Memory and Control Unit. It has 6 basic instructions to get a brief idea about how a processor works. Instructions are ADD, ADDI, SUB, AND, OR, BEQ and JUMP.

\section{Specifications of implemented Processor}
\begin{itemize}
\item 32-bit Instruction length
\item 16-bit Data and Address length
\item 128KB Register with 16-bit word length
\item 128KB Memory with 16-bit word length
\item 1Kb Instruction Length
\item 16-bit ALU with ADD, SUB, AND, OR Operations
\end{itemize}

\section{Dependencies}
Following programs are needed to run the simulation.
\begin{description}
\item{\textbf{ModelSim Simulator}} is used to simulate the behavior of processor. A script file for ModelSim will be generated by Python Program which will load Instructions in Instruction Register.
\item{\textbf{Python 2.7}} is used to create script file for ModelSim Simulator by taking instructions as Input.
\end{description}


\section{How to use}
\begin{itemize}
\item Enter Instructions of you program using Python Program \texttt{Mnemonics.py}
\item Enter \texttt{LOAD} to generate script file for ModelSim
\item Enter name of Script to save.
\item Open ModelSim and change your current directory to Script/
\item Enter \texttt{do <filename>.fdo}
This will initialize IR of Processor with instructions added in Python Program.
\item Click \texttt{Run} or enter command \texttt{run} in Transcript of ModelSim
\end{itemize}

\section{Instruction Set}
Will be updated soon.

\end{document}